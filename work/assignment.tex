% Options for packages loaded elsewhere
\PassOptionsToPackage{unicode}{hyperref}
\PassOptionsToPackage{hyphens}{url}
%
\documentclass[
]{article}
\usepackage{amsmath,amssymb}
\usepackage{lmodern}
\usepackage{ifxetex,ifluatex}
\ifnum 0\ifxetex 1\fi\ifluatex 1\fi=0 % if pdftex
  \usepackage[T1]{fontenc}
  \usepackage[utf8]{inputenc}
  \usepackage{textcomp} % provide euro and other symbols
\else % if luatex or xetex
  \usepackage{unicode-math}
  \defaultfontfeatures{Scale=MatchLowercase}
  \defaultfontfeatures[\rmfamily]{Ligatures=TeX,Scale=1}
\fi
% Use upquote if available, for straight quotes in verbatim environments
\IfFileExists{upquote.sty}{\usepackage{upquote}}{}
\IfFileExists{microtype.sty}{% use microtype if available
  \usepackage[]{microtype}
  \UseMicrotypeSet[protrusion]{basicmath} % disable protrusion for tt fonts
}{}
\makeatletter
\@ifundefined{KOMAClassName}{% if non-KOMA class
  \IfFileExists{parskip.sty}{%
    \usepackage{parskip}
  }{% else
    \setlength{\parindent}{0pt}
    \setlength{\parskip}{6pt plus 2pt minus 1pt}}
}{% if KOMA class
  \KOMAoptions{parskip=half}}
\makeatother
\usepackage{xcolor}
\IfFileExists{xurl.sty}{\usepackage{xurl}}{} % add URL line breaks if available
\IfFileExists{bookmark.sty}{\usepackage{bookmark}}{\usepackage{hyperref}}
\hypersetup{
  pdftitle={Lab 2: Regression to Study the Spread of Covid-19},
  pdfauthor={w203: Statistics for Data Science},
  hidelinks,
  pdfcreator={LaTeX via pandoc}}
\urlstyle{same} % disable monospaced font for URLs
\usepackage[margin=1in]{geometry}
\usepackage{longtable,booktabs,array}
\usepackage{calc} % for calculating minipage widths
% Correct order of tables after \paragraph or \subparagraph
\usepackage{etoolbox}
\makeatletter
\patchcmd\longtable{\par}{\if@noskipsec\mbox{}\fi\par}{}{}
\makeatother
% Allow footnotes in longtable head/foot
\IfFileExists{footnotehyper.sty}{\usepackage{footnotehyper}}{\usepackage{footnote}}
\makesavenoteenv{longtable}
\usepackage{graphicx}
\makeatletter
\def\maxwidth{\ifdim\Gin@nat@width>\linewidth\linewidth\else\Gin@nat@width\fi}
\def\maxheight{\ifdim\Gin@nat@height>\textheight\textheight\else\Gin@nat@height\fi}
\makeatother
% Scale images if necessary, so that they will not overflow the page
% margins by default, and it is still possible to overwrite the defaults
% using explicit options in \includegraphics[width, height, ...]{}
\setkeys{Gin}{width=\maxwidth,height=\maxheight,keepaspectratio}
% Set default figure placement to htbp
\makeatletter
\def\fps@figure{htbp}
\makeatother
\setlength{\emergencystretch}{3em} % prevent overfull lines
\providecommand{\tightlist}{%
  \setlength{\itemsep}{0pt}\setlength{\parskip}{0pt}}
\setcounter{secnumdepth}{-\maxdimen} % remove section numbering
\ifluatex
  \usepackage{selnolig}  % disable illegal ligatures
\fi

\title{Lab 2: Regression to Study the Spread of Covid-19}
\author{w203: Statistics for Data Science}
\date{10/28/2020}

\begin{document}
\maketitle

\hypertarget{introduction}{%
\section{Introduction}\label{introduction}}

In this lab, you will apply what you are learning about linear
regression to study the spread of COVID-19. Your task is to select a
research question, then conduct a regression study to analyze it.

Your research question must focus attention on a specific measurement
goal. It is not enough to say ``we are looking for policies that help
stop COVID-19'' (That would be a fishing expedition). Instead, use your
introduction to motivate a specific effect for measurement.

Once you have a measurement goal, you will build a set of linear models
that are tailored to that goal, and display them in a well formatted
regression table. You will finally use your conclusion to distill
insights from your estimates.

This is a group assignment. Your live session instructor will coordinate
the formation of groups. We would like to encourage teams to focus on
using the lab as a way to learn how to work as a team of collaborating
data scientists on shared code; how to clean and organize data; and, how
to present work in a compelling way. As a result, we encourage teams to
allow individuals to take risks and be supportive in the face of
successes and failures. Create an opportunity for people who want to
improve a particular skill to do so -- this might be project
coordination, management of code through git, plotting, or any of the
many aspects that you'll work on. \emph{We hope that you can support and
learn from one another through this team-based project.}

\hypertarget{deliverables}{%
\section{Deliverables}\label{deliverables}}

\begin{longtable}[]{@{}lll@{}}
\toprule
Deliverable Name & Week Due & Grade Weight \\
\midrule
\endhead
\protect\hyperlink{research-proposal}{Research Proposal} & Week 12 &
10\% \\
\protect\hyperlink{within-team-review}{Within-Team Review} & Week 12 &
5\% \\
\protect\hyperlink{final-presentation}{Final Presentation} & Week 14 &
10\% \\
\protect\hyperlink{final-report}{Final Report} & Week 14 & 75\% \\
\bottomrule
\end{longtable}

\hypertarget{data}{%
\section{Data}\label{data}}

We recommend the following data sources:

\begin{itemize}
\tightlist
\item
  \href{www.tinyurl.com/statepolicies}{COVID-19 US State Policy
  Database} A database of state policy responses to the pandemic,
  compiled by researchers at the Boston University School of Public
  Health.
\item
  \href{https://www.google.com/covid19/mobility/}{COVID-19 Community
  Mobility Report} A Google dataset that includes state-level
  measurements of individual mobility
\item
  \href{https://data.census.gov/cedsci/table?q=ACS\&g=0100000US.04000.001\&tid=ACSDP1Y2019.DP05\&moe=false\&hidePreview=true}{The
  American Community Survey} A product from the US Census Bureau that
  contains state-level demographics and other indicators of general
  interest.
\end{itemize}

If you want to, you are allowed to add extra variables from other
sources. However, this is not necessary, and we will not assign any
bonus points to teams that derive unique data, as our focus is on
statistics and statistical writing.

All data must be compiled into state-by-state metrics for regression
analysis.

\hypertarget{final-project-components}{%
\section{Final Project Components}\label{final-project-components}}

\hypertarget{research-proposal}{%
\subsection{Research Proposal}\label{research-proposal}}

After a week of work, the project team will produce a one page
research-proposal that defines a research question, data sources, and a
plan of action.

The research question should be informed by an understanding of the
datasets that are available, and the variables that are available in
those data sets. The team will need to do enough preliminary work to
assess whether their research question is feasible. A motivated team
might form their research question, and begin to build a functioning
data pipeline as an investment in ongoing project success. This proposal
is intended to provide a structure for the team to have early
conversations; it will be graded credit/no credit for completeness
(i.e.~a reasonable effort by the team will receive full marks).

\textbf{This proposal is due in week 12. Only one submission per team is
required.}

\hypertarget{within-team-review}{%
\subsection{Within-Team Review}\label{within-team-review}}

Being an effective, supportive team member is a crucial part of data
science work. Your performance on this lab includes the role you play in
supporting your teammates. This includes being responsive, creating an
environment in which all members feel included, and above all treating
each other with respect. In line with this perspective, we will ask each
team member to write to two paragraphs to their instructor about the
progress they have made individually, and the team has made as a whole
toward completing their report. This self-assessment should:

\begin{itemize}
\tightlist
\item
  Reflect on the strengths and weaknesses of the team and the team's
  process. Where your collaboration has worked well, how will you work
  to ensure that these successful practices continue to be employed? If
  there are places where collaboration has been challenging, what can
  the team do jointly to improve?
\item
  If there are any individual performances that deserve special
  recognition, please let me know in this evaluation. As well, if there
  are any individual performances that require special attention, please
  also let me know. I will treat these reviews as confidential and will
  not take any action without first consulting you.
\end{itemize}

\textbf{This reflection is due in week 12, and requires one submission
per person.} You will submit this through Gradescope, and like all parts
of your educational record, this will be treated confidentially by the
instructional team.

\hypertarget{final-presentation}{%
\subsection{Final Presentation}\label{final-presentation}}

During the Unit 14 live session, each team will give a slide
presentation of their work to their classmates -- i.e.~collaborating
data scientists. This audience is generally aware of the project that
you're working on, but they will need to be reminded of the specific
research question that you are addressing. \textbf{The presentation
should be structured as 10 minutes of presentation and 5 minutes of
questions from our classmates.} We'd like to note that this is an
\emph{incredibly} limited amount of time to present. The materials that
you present should reflect these serious constraints!

\begin{enumerate}
\def\labelenumi{\arabic{enumi}.}
\tightlist
\item
  There should be no more than two slides -- probably one is best --
  that set-up your research problem and these slides should take no more
  than two minutes to present. (1 minute)
\item
  There should be at least one, and not more than two, slides that
  describe the variables that you're using in your models. These slides
  should cover the important features of the variables that you're
  using: how are they are measured, the unit of observation, plots of
  how these variables are distributed, and why this \emph{particular}
  variable appropriate to use to answer your research question. (3
  minutes)

  \begin{itemize}
  \tightlist
  \item
    Do not present R code, discuss data wrangling, or normality -
    details like this are best left to the full analysis.
  \end{itemize}
\item
  There should then be several slides that provide what you've learned
  from your models. If you show model results, you need to provide your
  audience with enough time to read and engage with these models; not
  flash past them. Any model you show will take at least one minute to
  talk about.
\end{enumerate}

Finally, a few more general thoughts:

\begin{itemize}
\tightlist
\item
  Practice your talk with a timer. We're going to be strict and end your
  talk 10 minutes in -- we have a mute button! :joy:
\item
  If you divide your talk with your teammates, practice your section
  with a timer so that you do not spill over into your teammates' time.
\item
  We \emph{strongly} recommend having no more than 5 or maybe 6 slides
  total.
\end{itemize}

\hypertarget{final-report}{%
\subsection{Final Report}\label{final-report}}

Your final deliverable is a written statistical analysis documenting
your findings. Please limit your submission to 8000 words, excluding
code cells and R output.

The exact format of your report is flexible, but it should include the
following elements.

\hypertarget{an-introduction}{%
\subsubsection{1. An Introduction}\label{an-introduction}}

Your introduction should present a research question and explain the
concept that you're attempting to measure and how it will be
operationalized. This section should pave the way for the body of the
report, preparing the reader to understand why the models are
constructed the way that they are. It is not enough to simply say ``We
are looking for policies that help against COVID.'' Your introduction
must do work for you, focusing the reader on a specific measurement
goal, making them care about it, and propelling the narrative forward.
This is also good time to put your work into context, discuss
cross-cutting issues, and assess the overall appropriateness of the
data.

\hypertarget{a-model-building-process}{%
\subsubsection{2. A Model Building
Process}\label{a-model-building-process}}

You will next build a set of models to investigate your research
question, documenting your decisions. Here are some things to keep in
mind during your model building process:

\begin{enumerate}
\def\labelenumi{\arabic{enumi}.}
\tightlist
\item
  \emph{What do you want to measure}? Make sure you identify one key
  variable (possibly more in rare cases) that will allow you to derive
  conclusions relevant to your research question, and include this
  variables in all model specifications.
\item
  Is your modeling goal one of description or explanation?
\item
  What
  \href{https://en.wikipedia.org/wiki/Dependent_and_independent_variables\#Statistics_synonyms}{covariates}
  help you achieve your modeling goals? What covariates are problematic,
  either due to \emph{collinearity}, or because they will absorb some of
  a causal effect you want to measure?
\item
  What \emph{transformations}, if any, should you apply to each
  variable? These transformations might reveal linearities in the data,
  make your results relevant, or help you meet model assumptions.
\item
  Are your choices supported by exploratory data analysis (\emph{EDA})?
  You will likely start with some general EDA to \emph{detect anomalies}
  (missing values, top-coded variables, etc.). From then on, your EDA
  should be interspersed with your model building. Use visual tools to
  \emph{guide} your decisions. You can also leverage statistical
  \emph{tests} to help assess whether variables, or groups of variables,
  are improving model fit.
\end{enumerate}

At the same time, it is important to remember that you are not trying to
create one perfect model. You will create several specifications, giving
the reader a sense of how robust (or sensitive) your results are to
modeling choices, and to show that you're not just cherry-picking the
specification that leads to the largest effects.

At a minimum, you should include the following three specifications:

\begin{enumerate}
\def\labelenumi{\arabic{enumi}.}
\tightlist
\item
  \textbf{Limited Model}: A model that includes \emph{only the key
  variable} you want to measure and nothing (or almost nothing) else.
  This variables might be transformed, as determined by your EDA, but
  the model should include the absolute minimum number of covariates
  (perhaps one, or at most two-three, covariates if they are so crucial
  that it would be unreasonable to omit them).
\item
  \textbf{Model Two}: A model that includes \emph{key explanatory
  variables and covariates that you believe advance your modeling} goals
  without introducing too much multicollinearity or causing other
  issues. This model should strike a balance between accuracy and
  parsimony and reflect your best understanding of the relationships
  among key variables.
\item
  \textbf{Model Three}: A model that includes the \emph{previous
  covariates, and many other covariates}, erring on the side of
  inclusion. A key purpose of this model is to evaluate how parameters
  of interest change (if at all) when additional, potentially colinear
  variables are included in the model specification.
\end{enumerate}

Although the models have different emphases, each one must still be a
reasonable choice given your modeling goals. The idea is to choose
models that encircle the space of reasonable modeling choices, and to
give an overall understanding of how these choices impact results.

\hypertarget{a-regression-table}{%
\subsubsection{3. A Regression Table}\label{a-regression-table}}

You should display all of your model specifications in a regression
table, using a package like
\href{https://cran.r-project.org/web/packages/stargazer/vignettes/stargazer.pdf}{\texttt{stargazer}}
to format your output. It should be easy for the reader to find the
coefficients that represent key effects at the top of the regression
table, and scan horizontally to see how they change from specification
to specification. Make sure that you display the most appropriate
standard errors in your table, along with significance stars.

In your text, comment on both \emph{statistical significance and
practical significance}. You may want to include statistical tests
besides the standard t-tests for regression coefficients.

\hypertarget{limitations-of-your-model}{%
\subsubsection{4. Limitations of your
Model}\label{limitations-of-your-model}}

As a team, evaluate all of the CLM assumptions that must hold for your
model. However, do not report an exhaustive examination all 5 CLM
assumption. Instead, bring forward only those assumptions that you think
pose significant problems for your analysis. For each problem that you
identify, describe the statistical consequences. If you are able to
identify any strategies to mitigate the consequences, explain these
strategies.

Note that you may need to change your model specifications in response
to violations of the CLM.

\hypertarget{discussion-of-omitted-variables}{%
\subsubsection{5. Discussion of Omitted
Variables}\label{discussion-of-omitted-variables}}

If the team has taken up an explanatory (i.e.~causal) question to
evaluate, then identify what you think are the 5 most important
\emph{omitted variables} that bias results you care about. For each
variable, you should \emph{reason about the direction of bias} caused by
omitting this variable. If you can argue whether the bias is large or
small, that is even better. State whether you have any variables
available that may proxy (even imperfectly) for the omitted variable.
Pay particular attention to whether each omitted variable bias is
\emph{towards zero or away from zero}. You will use this information to
judge whether the effects you find are likely to be real, or whether
they might be entirely an artifact of omitted variable bias.

\hypertarget{conclusion}{%
\subsubsection{6. Conclusion}\label{conclusion}}

Make sure that you end your report with a discussion that distills key
insights from your estimates and addresses your research question.

\hypertarget{rubric-for-evaluation}{%
\subsection{Rubric for Evaluation}\label{rubric-for-evaluation}}

You may use the following, loosely structured rubric to guide your
writing.

\begin{itemize}
\item
  \textbf{Introduction.} Is the introduction clear? Is the research
  question specific and well defined? Does the introduction motivate a
  specific concept to be measured and explain how it will be
  operationalized. Does it do a good job of preparing the reader to
  understand the model specifications?
\item
  \textbf{The Initial Data Loading and Cleaning.} Did the team notice
  any anomalous values? Is there a sufficient justification for any data
  points that are removed? Did the report note any coding features that
  affect the meaning of variables (e.g.~top-coding or bottom-coding)?
  Overall, does the report demonstrate a thorough understanding of the
  data? Does the report convey this understand to its reader -- can the
  reader, through reading this report, come to the same understanding
  that the team has come to?
\item
  \textbf{The Model Building Process.} Overall, is each step in the
  model building process supported by EDA? Is the outcome variable
  appropriate? Did the team clearly state why they chose these
  explanatory variables, does this explanation make sense in term of
  their research question? Did the team consider available variable
  transformations and select them with an eye towards model plausibility
  and interpretability? Are transformations used to expose linear
  relationships in scatterplots? Is there enough explanation in the text
  to understand the meaning of each visualization?
\item
  \textbf{Regression Models:}

  \begin{itemize}
  \tightlist
  \item
    \textbf{Base Model.} Does this model only include key explanatory
    variables? Do the variables make sense given the measurement goals?
    Did the team apply reasonable transformations to these variables, to
    capture the nature of the relationships? Does the team write about
    this model in prose in a way that is appropriate?
  \item
    \textbf{Second Model.} Does this model represent a balanced
    approach, including variables that advance modeling goals without
    causing major issues? Does the model succeed in reducing standard
    errors of the key variables compared to the base model? Does it
    capture major non-linearities in the joint distribution of the
    variables? Does the team write about this model in prose in a way
    that is appropriate?
  \item
    \textbf{Third Model.} Does this model represent a maximalist
    approach, erring on the side of including most variables? Is it
    still a reasonable model? Are there any variables that are outcomes,
    and should therefore still be excluded? Is there too much
    colinearity, to the point that the key causal effects cannot be
    measured? Does this team write about this model in prose in a way
    that is appropriate?
  \end{itemize}
\item
  \textbf{A Regression Table.} Are the model specifications properly
  chosen to outline the boundary of reasonable choices? Is it easy to
  find key coefficients in the regression table? Does the text include a
  discussion of practical significance for key effects?
\item
  \textbf{Plots, Figures, and Tables} Do the plots, figures and tables
  that the team has chosen to include successfully move forward the
  argument that they are making? Has the team chosen the most effective
  method (a table or a chart) to display their evidence? Is that table
  or chart the most communicative it could be? Is every plot, figure,
  and table that is included in the report referenced in the narrative
  argument?
\item
  \textbf{Assessment of the CLM.} Has the team presented a sober
  assessment of the CLM assumptions that might be problematic for their
  model? Have they presented their analysis about the consequences of
  these problems (including random sampling) for the models they
  estimate? Did they use visual tools or statistical tests, as
  appropriate? Did they respond appropriately to any violations?
\item
  \textbf{An Omitted Variables Discussion.} Did the report miss any
  important sources of omitted variable bias? Are the estimated
  directions of bias correct? Was their explanation clear? Is the
  discussion connected to whether the key effects are real or whether
  they may be solely an artifact of omitted variable bias?
\item
  \textbf{Conclusion.} Does the conclusion address the research
  question? Does it raise interesting points beyond numerical estimates?
  Does it place relevant context around the results?
\item
  Are there any other errors, faulty logic, unclear or unpersuasive
  writing, or other elements that leave you less convinced by the
  conclusions?
\end{itemize}

\end{document}
